\documentclass{beamer}
\usetheme{default}
\usepackage[italian]{babel}
%\usepackage{t1enc}
\usepackage{pgfpages}
\usepackage{listings}
%\pgfpagelayout{2 on 1}[a4paper]

%\usepackage{../../common/espacs}
\lstset
{
  language=[ISO]C++,                       % The default language
  basicstyle=\sf,                          % The basic style
  keywordstyle=\color{blue}\bfseries,      % Set keyword style
  commentstyle=\color{darkgreen}\itshape,  % Set comment style
  extendedchars=true                       % Allow extended characters
}

\setbeamercovered{transparent}

\begin{document}

%---------------------------------------------------------------------------------

\begin{frame}[fragile]

    Breve ripasso e introduzione alla compilazione di un codice C++.

\end{frame}

%---------------------------------------------------------------------------------

\begin{frame}[fragile]

    \frametitle{Come compilare un codice C++}

    Consideriamo il seguente programma \\
    \vspace{.3cm}

    \lstset{basicstyle=\scriptsize\sf}
    \lstinputlisting{./main.cpp}
    \lstset{basicstyle=\sf}

    \vspace{.3cm}
    salvato nel file \verb1main.cpp1.

\end{frame}

%---------------------------------------------------------------------------------

\begin{frame}[fragile]

    \frametitle{Come compilare un codice C++}

    Il codice pu\`o essere compilato ed eseguito mediante i seguenti comandi

    \begin{verbatim}
        g++ -c main.cpp
        g++ -o main main.o
        ./main
    \end{verbatim}

    \vspace{.3cm}
    Ottenendo
    \begin{verbatim}
        Hello World!
    \end{verbatim}

\end{frame}

%---------------------------------------------------------------------------------

\begin{frame}[fragile]

    \frametitle{Come compilare un codice C++}

    Un modo alternativo per compilare il codice, che permette di non generare il
    file oggetto \verb1main.cpp1, \`e mediante i seguenti comandi
    \begin{verbatim}
        g++ -o main main.cpp
        ./main
    \end{verbatim}

    \vspace{.3cm}
    In questo caso la fase di compilazione e quella di linking vengono fatte
    in sequenza e attraverso un unico comando.

\end{frame}

%---------------------------------------------------------------------------------

\begin{frame}[fragile]

    \frametitle{Come compilare un codice C++}

    Esistono numerose flag per il compilatore che permettono di ottenere
    diversi comportamenti

    \begin{itemize}
        \item \verb1-Wall1 per segnalare eventuali warnings;
        \item \verb1-O1 per ottenere il codice pi\`u efficiente;
        \item \verb1-g1 per ottenere il codice utile per il debugging.
    \end{itemize}

\end{frame}

%---------------------------------------------------------------------------------

\begin{frame}[fragile]

    \frametitle{Come compilare un codice C++}

    La flag \verb1-Wall1 permette di segnalare eventuali warnings.
    Se utilizziamo tale flag in fase di compilazione per il programma \verb1main.cpp1

    \begin{verbatim}
        g++ -o main main.cpp -Wall
    \end{verbatim}

    \vspace{.3cm}
    Ottenendo
    \begin{verbatim}
        main.cpp: In function 'int main()':
        main.cpp:7: warning: unused variable 'i'
    \end{verbatim}

\end{frame}

%---------------------------------------------------------------------------------

\begin{frame}[fragile]

    \frametitle{Come compilare un codice C++}

    Il flag \verb1-O1 seguito da un numero compreso tra $1$ e $3$, permette di
    ottenere un codice pi\`u efficiente a scapito del tempo di compilazione e
    dell'occupazione fisica dell'eseguibile.\\

    \vspace{.3cm}
    In particolare

    \begin{itemize}

        \item \verb2-O12 il compilatore cerca di ridurre le dimensioni del codice
            e il tempo di esecuzione, senza eseguire alcuna ottimizzazione che
            richiede una grande quantit� di tempo di compilazione.

        \item \verb1-O21 il compilatore esegue quasi tutte le ottimizzazioni che
            comportano un buon compromesso spazio occupato-velocit\`a di esecuzione.
            Rispetto al \verb2-O12, questa opzione aumenta sia il tempo di
            compilazione che le prestazioni del codice generato.

        \item \verb1-031 livello di ottimizzazione massimo, il compilatore
            attiva tutte le procedure per ottenere un codice il pi\`u veloce
            possibile.

    \end{itemize}

\end{frame}

%---------------------------------------------------------------------------------

\begin{frame}[fragile]

    \frametitle{Come compilare un codice C++}

    Se utilizziamo tale flag in fase di compilazione per il programma
    \verb1main.cpp1

    \begin{verbatim}
        g++ -o main main.cpp -O2
    \end{verbatim}

    \vspace{.3cm}
    \`E possibile utilizzare pi\`u di una flag in contemporanea, l'ordine non conta
    \begin{verbatim}
        g++ -o main main.cpp -Wall -O2
    \end{verbatim}

\end{frame}

%---------------------------------------------------------------------------------

\end{document}
