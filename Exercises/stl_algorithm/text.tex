\section*{Es.}
Per ogni punto creare un programma differente.
\begin{enumerate}
	\item Creare un vettore di \cpp{complex} e ordinarlo, utilizzando l'algoritmo della STL \cpp{sort}, 
	in maniera decrescente rispetto al valore assoluto.
	\item Creare un vettore di interi avente diverse occorrenze del numero 55. Utilizzare l'algoritmo
	della STL \cpp{find} per trovare il terzo elemento di valore 55.
	\item Creare una lista di \cpp{float} e contare il numero di elementi il cui valore assoluto \`e minore
	di 4.5. Utilizzare l'algoritmo della STL \cpp{count_if}.
	\item Creare un insieme di interi e trovare la somma di tutti gli elementi, ciascuno moltiplicato per 3.
	Utilizzare l'algoritmo della STL \cpp{accumulate} sia per la somma che per la moltiplicazione.
	\item Creare due insiemi di interi e trovare la loro intersezione e la loro unione, utilizzando gli
	algoritmi della STL \cpp{set_union} e \cpp{set_intersection}.
	\item Creare una mappa che prende come chiave un intero e come valore un puntatore a funzione, che 
	rappresenta una funzione da $\mathbb{R}^3 \rightarrow \mathbb{R}$. Inserire nella mappa 
	le seguenti due funzioni, di chiave rispettivamente $1$ e $5$
	\begin{align*}
		f_1(x,y,z) &= x+2y-z^2\,,\\
		f_5(x,y,z) &= 2y-6\,.
	\end{align*}
	Estrarre quindi dalla mappa la funzione associata alla chiave $5$ e valutarla nel punto $(1,2,3)$.
\end{enumerate}
