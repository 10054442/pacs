\section*{Esercizio}

Si vuole integrare nella programma \texttt{AllDynamic} la possibilit\`a di
gestire anche le integrande attraverso le factory. In particolare si vuole
collezionare una serie di librerie dinamiche contenenti le integrande, anche
definite tramite funtori. Il caricamento delle librerie, il cui numero \`e a
priori incognito, deve rimanere dinamico e la scelta dell'integranda deve poter
essere fatta ancora tramite il file di dati gestito da GetPot.
Ogni libreria che contiene le funzioni deve essere indipendente dalle altre e
deve poter autonomamente registrare nella factory i propri dati. In questo modo
all'aggiunta di una nuova libreria non bisogner\`a fare altro che inserire il
nome della nuovo libreria nel file di dati, i suoi oggetti verrano ad essere
automaticamente gestiti della factory.

Non si vuole modificare la classe \texttt{Factory} in alcun modo.
I punti per ottenere quanto richiesto sono i seguenti, facendo attenzione di
modificare opportunamente in \texttt{Makefile} per poter gestire le nuove
funzionalit\`a.

\begin{enumerate}

    \item Non utilizzare pi\`u la classe \texttt{udfHandler}, al suo posto
    verr\`a utilizzata la factory.

    \item Utilizzare il function wrapper della standard library per uniformare
    il tipo che dovr\`a essere contenuto nelle factory.

    \item Creare un nuova nuova classe proxy che servir\`a per la registrazione
    delle funzioni. Capire perch\'e \`e utile scorporare il metodo builder dal proxy
    vero e proprio.

    \item Non \`e quindi pi\`u utile inserire le funzioni all'interno degli
    \texttt{extern "C"}. Creare quindi un file per la registrazione delle
    funzioni utilizzando il proxy introdotto.

    \item Creare un funtore e associargli la registrazione.

\end{enumerate}

Un'aggiunta per rendere pi\`u robusta la gestione delle regole di quadratura \`e
determinare staticamente se il parametro \texttt{AbstractProduct\_type},
definito nella factory, \`e genitore di \texttt{ConcreteProduct}. Inoltre il
metodo statico \texttt{Build} deve essere poter convertito nel tipo
corrispettivo definito nella factory. Utilizzare i \texttt{traits} definiti
nella standard libray e lo \texttt{static\_assert}.
