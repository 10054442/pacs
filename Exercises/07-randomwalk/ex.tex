\section*{Exercise 1}

Implement a random walk in one dimension, using the following tips:
\begin{itemize}
  \item use a \cpp{map} to store the distribution of particles.
  \item use as starting configuration a Dirac delta --- approximated with
  $10000$ particles located at zero (use a much smaller value until the code
  is functioning properly).
  \item consider that the particles move by 1 or -1 at each time step,
  \item use a \cpp{discrete_distribution} to emulate the random motion, with
  equal probability of moving forwards or backwards.
  \item simulate 20 time steps.
  \ item print out the distribution to screen and to file, in order to plot it
  in \texttt{gnuplot}.
  \item plot also the fundamental solution in gnuplot to compare the result,
  remembering that $D = \frac{h^2}{2t}$.
  \item check what happens if the distrubution is biased towards one direction,
  or if the particle is allowed to remain in the same position, without	moving.
\end{itemize}

\section*{Exercise 2}

Implement a class that performs MonteCarlo (MC) integration, using a suitable
probability distribution.

