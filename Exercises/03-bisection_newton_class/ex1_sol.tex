\section*{Soluzioni}

\begin{enumerate}

    \item Per rendere riutilizzabile il codice, si \`e scelto di raggruppare le
        interfacce delle classi create in \emph{header file}, mentre le
        implementazioni si trovano nei corrispondenti \emph{source
        file}. Per rendere disponibile in tutti gli \emph{header file} le
        definizioni generali, quali \cpp{enum} e \cpp{typedef}, \`e meglio
        raggrupparle nel seguente \emph{header file}:

        \lstset{basicstyle=\scriptsize\sf}
            \lstinputlisting{./es/1/src/rootfinding.hpp}
        \lstset{basicstyle=\sf}

        In questo caso, i nomi definiti da \cpp{rootfinding.hpp} sono
        esportati nel \cpp{namespace} \cpp{Rootfinding}.

        Si riportano di seguito i listati degli \emph{header file} e dei
        \texttt{source files} che contengono le definizioni delle classi
        \cpp{Bisection} e \cpp{Newton}.
        % Bisection
        \lstset{basicstyle=\scriptsize\sf}
            \lstinputlisting[caption=Interfaccia della classe \cpp{Bisection}]
                {es/1/src/bisection.hpp}
        \lstset{basicstyle=\sf}
        Si possono notare i nomi degli attributi privati all'interno della classe,
        sono tutti della forma \cpp{M_nomeattributo}.
        \`E infatti buona norma utilizzare questo tipo di sintassi per rendere
        facilmente riconoscibile, all'interno dell'implementazione di metodi della classe,
        gli attributi della classe dalle variabili create localmente in ciascun metodo.

        In metodo \cpp{converged} \`e una funzione di tipo \cpp{inline}, ovvero \`e una
        direttiva del C++ usata per indicare al compilatore che deve eseguire l'espansione
        inline della funzione. In altre parole, il compilatore tenter\`a
        (\`e una richiesta rivolta al compilatore, non un obbligo) ed inserire il corpo
        completo della funzione ogni qual volta, nel codice, viene chiamata tale funzione.
        Come si vede dal codice, \`e possibile separare la dichiarazione della funzione
        \cpp{inline} dalla sua implementazione. Nel caso in cui il metodo \cpp{inline} \`e
        implementato all'interno della classe la direttiva \cpp{inline} \`e opzionale, in
        quanto il compilatore interpreta tali metodi come \cpp{inline}. Se tuttavia il metodo
        \`e implementato al di fuori della classe \`e necessaria la direttiva \cpp{inline} e
        la funzione deve essere implementata nell'\textit{header file}. Si consiglia di
        inserire all'interno della classe solamente metodi molto brevi.
        Inoltre la direttiva \cpp{inline} \`e applicabile anche alle \textit{free function},
        ovvero a quelle funzioni non metodi di una classe. L'utilizzo e l'effetto sono analoghi.
        \lstset{basicstyle=\scriptsize\sf}
            \lstinputlisting[caption=Implementazione della classe \cpp{Bisection}]
                {es/1/src/bisection.cpp}
        \lstset{basicstyle=\sf}
        All'interno del costruttore della classe \`e possibile chiamare esplicitamente
        i costruttori degli attributi della classe.
        Nel caso considerato verranno creati gli oggetti \cpp{M_tol}, \cpp{M_maxit} e
        \cpp{M_check} direttamente con i valori \cpp{tol}, \cpp{maxit} e \cpp{check},
        rispettivamente.

        I commenti alla classe \cpp{Newton} sono analoghi a quelli fatti per la
        classe \cpp{Bisection}. Riportiamo i listati contenenti i codici.
        % Newton
        \lstset{basicstyle=\scriptsize\sf}
            \lstinputlisting[caption=Interfaccia della classe \cpp{Newton}]
                {es/1/src/newton.hpp}
            \lstinputlisting[caption=Implementazione della classe \cpp{Newton}]
                {es/1/src/newton.cpp}
        \lstset{basicstyle=\sf}


    \item Riportiamo qui di seguito il listato per la classe \cpp{Robust}, divisa anch'essa
        in \emph{header file} e \emph{source file}.
        % Robust
        \lstset{basicstyle=\scriptsize\sf}
            \lstinputlisting[caption=Interfaccia della classe \cpp{Robust}]{es/1/src/robust.hpp}
            \lstinputlisting[caption=Implementazione della classe \cpp{Robust}]{es/1/src/robust.cpp}
        \lstset{basicstyle=\sf}

        All'interno della struttura \cpp{robust} sono stati definiti i tipi
        per il metodo globalmente convergente utilizzato per trovare
        l'approssimazione grossolana della radice (\cpp{coarseT}) e per il
        metodo di alto ordine utilizzato per raffinare la stima
        (\cpp{fineT}). Ci\`o pu\`o essere utile qualora si desiderasse
        cambiare uno dei due metodi, a patto, naturalmente, che la classe che
        implementa il nuovo metodo abbia un'interfaccia compatibile con quella
        del metodo che sostituisce. Si provi, per esercizio, a scrivere
        un'implementazione del metodo di Newton modificato e ad utilizzarlo
        nella struttura \cpp{robust}. Si vedr\`a nel seguito che questa
        tecnica pu\`o essere ulteriormente raffinata parametrizzando la classe
        \cpp{Robust} rispetto ai due tipo \cpp{coarseT} e \cpp{fineT}
        utilizzando la direttiva \cpp{template}.
        Il grado di approssimazione dello zero richiesto al metodo grossolano
        \`e definito come il prodotto della tolleranza finale \cpp{tol} per il
        coefficiente \cpp{M\_cfratio} (\emph{coarse-to-fine ratio}).
        I membri \emph{accessor} permettono di accedere accedere alle
        propriet\`a non modificabili dall'utente tramite un membro \cpp{const}.

        Il file \cpp{Makefile}, che permette la  compilazione di \cpp{bisection},
        \cpp{newton} e \cpp{robust}, produce una libreria statica
        \cpp{librootfinding.a}. Una volta preparato il file di libreria, qualsiasi
        codice che utilizzi questi metodi, come quello proposto al \emph{Punto 3}, pu\`o
        essere compilato passando al compilatore il nome della libreria (tramite il
        parametro \texttt{-lrootfinding}) e la sua posizione (\texttt{-Llib} in questo esempio).

        Le strutture delle classi \cpp{bisection},
        \cpp{newton} e \cpp{robust} hanno molte propriet\`a analoghe. Replicare
        la definizione strutture simili \`e non solo inelegante, ma rischioso
        qualora divenga importante %in un altro punto del codice
        che le classi che implementano i diversi metodi abbiano un'interfaccia
        omogenea. Una soluzione migliore consiste nel creare una classe base
        \cpp{IterativeMethod} da cui derivare le implementazioni di ciascun
        metodo.


    \item Si riporta di seguito il codice del \emph{main program} per la
        soluzione del terzo punto. L'implementazione cui si fa riferimento
        \`e quella relativa al listato riportato ai punti precedente.
        \lstset{basicstyle=\scriptsize\sf}
            \lstinputlisting{./es/1/bn.hpp}
        \lstset{basicstyle=\sf}
        In questo caso le classi \cpp{Bisection}, \cpp{Newton} e \cpp{Robust} sono
        salvate nella cartella \verb1./src/1.
        \lstset{basicstyle=\scriptsize\sf}
            \lstinputlisting{./es/1/bn.cpp}
        \lstset{basicstyle=\sf}
        All'interno del \verb1main1 si noti l'utilizzo sia di \cpp{using namespace std}
        che di \cpp{using namespace RootFinding}. Utile per accedere agevolmente a tutti
        gli oggetti e funzioni dichiarate in \cpp{std} e \cpp{RootFinding}.

    \item  Si fornisce la soluzione relativa alla struttura \cpp{Robust}. Il
        sovraccaricamento dell'\cpp{operator<<} negli altri casi \`e del tutto
        analogo.
        \lstset{basicstyle=\scriptsize\sf}
            \lstinputlisting[linerange={15-30}]{./es/2/src/robust.cpp}
        \lstset{basicstyle=\sf}

        Si noti che la funzione che sovraccarica l'operatore di scorrimento
        accede ai membri privati della classe. Per questa ragione \`e
        necessario specificare che si tratta di una \emph{friend function}
        aggiungendo l'istruzione seguente all'interfaccia della classe:
        \lstset{basicstyle=\scriptsize\sf}
            \lstinputlisting[linerange=46-47]{./es/1.5/src/robust.hpp}
        \lstset{basicstyle=\sf}

        Per testare il codice scritto si pu\`o utilizzare il seguente \emph{main program}:
        %
        \lstset{basicstyle=\scriptsize\sf}
            \lstinputlisting{./es/1.5/bn.cpp}
        \lstset{basicstyle=\sf}
        %
        Compilando ed eseguendo il programma si ottiene il seguente \emph{output}:
        \begin{verbatim}
0.707107
* Robust Method *
Tolerance           :   1e-06
Max # of iterations :   100
Convergence check   :   increment
# of iterations (C) :   3
# of iterations (F) :   4
C-to-F tol ratio    :   200000
        \end{verbatim}

\end{enumerate}
