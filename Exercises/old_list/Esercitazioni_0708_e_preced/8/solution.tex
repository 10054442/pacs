\newpage

\section*{Soluzione}

\subsection*{Es. 1}

Una possibile soluzione consiste nell'implementare una classe
\cpp{QuadratureBase} che rappresenti un oggetto generico
\emph{integratore numerico}.

\lstset{basicstyle=\scriptsize\sf}
\lstinputlisting[caption=Dichiarazione della classe
\cpp{QuadratureBase}, linerange={1-49}]{es/1/inherit1/src/numerical_integration.hpp}
\lstset{basicstyle=\sf}

Da questa pu\`o essere derivata pubblicamente la classe \cpp{Simpson}
che risolve il primo punto dell'esercizio.

\lstset{basicstyle=\scriptsize\sf}
\lstinputlisting[caption=Dichiarazione della classe
\cpp{Simpson}, linerange={51-71}]{es/1/inherit1/src/numerical_integration.hpp}
\lstset{basicstyle=\sf}

Allo stesso modo si definisce la classe \cpp{MidPoint}, per il secondo
punto dell'esercizio.

\lstset{basicstyle=\scriptsize\sf}
\lstinputlisting[caption=Dichiarazione della classe
\cpp{MidPoint}, linerange={73-97}]{es/1/inherit1/src/numerical_integration.hpp}
\lstset{basicstyle=\sf}

Si noti che la classe \cpp{QuadratureBase} \`e \emph{astratta},
poich\`e il metodo virtuale \cpp{apply} \`e definito uguale a
\cpp{0}. Il suo costruttore pu\`o essere invocato soltanto dalle
classi derivate, e si occupa dell'inizializzazione della struttura
dati su cui lavora l'integratore (una lista di nodi).

Le classi derivate forniscono l'implementazione del metodo
\cpp{apply}, applicando i metodi di integrazione.

\lstset{basicstyle=\scriptsize\sf}
\lstinputlisting[caption=Definizioni delle classi per integrazione
numerica]{es/1/inherit1/src/numerical_integration.cpp}
\lstset{basicstyle=\sf}

Si pu\`o pensare di organizzare il codice in un altro modo,
interpretando la \emph{regola di quadratura} come un oggetto in grado
di calcolare l'integrale di una funzione su un intervallo di
riferimento. In questo modo l'oggetto \emph{integratore} riceve la
\emph{regola di quadratura} come parametro, invocandone le funzionalit\`a.

\lstset{basicstyle=\scriptsize\sf}
\lstinputlisting[caption=Una strategia alternativa: definire oggetti
di classe \emph{regola di quadratura}. Dichiarazioni.,linerange={14-82}]{es/1/inherit2/src/numerical_rule.hpp}
\lstinputlisting[caption=Una strategia alternativa: definire oggetti
di classe \emph{regola di quadratura}. Definizioni.]{es/1/inherit2/src/numerical_rule.cpp}
\lstset{basicstyle=\sf}

In questo modo il codice risulta pi\`u flessibile: ciascun oggetto
esporta un insieme minimale di operazioni che vengono assemblate con
un approccio modulare. La classe \cpp{Quadrature} in questo caso si
occupa di invocare una \cpp{QuadratureRule} su ciascun sottointervallo
dell'intervallo di integrazione. L'introduzione di nuove classi
derivate che
implementino regole di quadratura � immediata.
\lstset{basicstyle=\scriptsize\sf}
\lstinputlisting[caption=Dichiarazione della classe
\cpp{Quadrature},linerange={20-53}]{es/1/inherit2/src/numerical_integration.hpp} 
\lstinputlisting[caption=Definizione della classe
\cpp{Quadrature}]{es/1/inherit2/src/numerical_integration.cpp}
\lstset{basicstyle=\sf}

Il prezzo da pagare \`e che la classe \cpp{Quadrature} richiede un
riferimento ad un oggetto di tipo \cpp{QuadratureRule}: nella
soluzione proposta si tratta di una variabile privata. In generale
questo appesantisce l'interfaccia di \cpp{Quadrature} e rischia di
rendere il codice meno leggibile (nel caso in cui non sia chiaro cosa
aspettarsi da oggetti di tipo \cpp{QuadratureRule}).

Questo stesso svantaggio si riscontra nell'implementazione seguente,
nella quale la \emph{regola di quadratura} \`e un tipo generico,
rappresentato da un parametro \cpp{template} per la classe \cpp{Quadrature}:

\lstset{basicstyle=\scriptsize\sf}
\lstinputlisting[caption=Programmazione generica rispetto ai tipi:
classe \cpp{template} \cpp{Quadrature}]{es/1/template/src/numerical_integration.hpp}
\lstset{basicstyle=\sf}

D'altra parte in questo caso non c'\`e alcun vincolo
sull'implementazione del parametro \cpp{template}: va rispettata unicamente la
\emph{policy} del tipo \cpp{QR}. La classe
\cpp{QuadratureRule} soddisfa questa \emph{policy}, esportando i metodi
pubblici \cpp{node} e \cpp{weight}.

 Si noti che la programmazione
generica rispetto ai tipi rende potenzialmente pi� costosa
l'introduzione di nuove classi per le regole di quadratura, poich�
rende necessaria la ricompilazione dell'intero codice in cui compaia
il parametro \cpp{template}
(\emph{polimorfismo statico}). Lo sfruttamento del polimorfismo
\emph{dinamico} nella
programmazione orientata agli oggetti, per contro, consente in linea
di principio di estendere il software semplicemente creando nuove unit� di
compilazione.

Il \texttt{main program} invoca gli oggetti che descrivono i diversi
integratori. Si noti l'utilizzo del software \texttt{GetPot}: la variabile
\cpp{cl} contiene un database, costruito a partire dalle variabili
\cpp{argc} e \cpp{argv}, che consente di interpretare la lista di
parametri forniti da riga di comando:
\begin{lstlisting}
./main_integration a=0 b=10 nint=100
\end{lstlisting}
produce l'inizializzazione delle variabili \cpp{double} \cpp{a=0} e
\cpp{b=10} e della variabile \cpp{int} \cpp{nint=100}.

\lstset{basicstyle=\scriptsize\sf}
\lstinputlisting[caption=\texttt{main program} per il calcolo
dell'integrale, linerange={1-49}]{es/1/inherit1/main_integration.cpp}
\lstset{basicstyle=\sf}

Nel caso in cui si definiscano le classi \cpp{Simpson} e
\cpp{MidPoint} come derivate da \cpp{QuadratureBase}, l'invocazione
nel \texttt{main program} � la seguente:
\lstset{basicstyle=\scriptsize\sf}
\lstinputlisting[caption=Integratori come classi derivate da
\cpp{QuadratureBase}, linerange={49-57}]{es/1/inherit1/main_integration.cpp} 
\lstset{basicstyle=\sf}
Il codice riportato contiene degli esempi di \emph{polimorfismo dinamico}:
puntatori e riferimenti a oggetti di classe base possono riferirsi a
oggetti di classe derivata.

Quando invece la regola di quadratura � descritta da una classe
\cpp{QuadratureRule}, il codice va modificato come segue:
\lstset{basicstyle=\scriptsize\sf}
\lstinputlisting[caption=Regole di integrazione come classi derivate da
\cpp{QuadratureRule}, linerange={34-37,47-54}]{es/1/inherit2/main_integration.cpp} 
\lstset{basicstyle=\sf}

Infine la classe \cpp{template} \cpp{Quadrature} va invocata con la
lista esplicita dei suoi parametri \cpp{template}:
\lstset{basicstyle=\scriptsize\sf}
\lstinputlisting[caption=Classe \cpp{template}
\cpp{Quadrature}, linerange={45-58}]{es/1/template/main_integration.cpp} 
\lstset{basicstyle=\sf}

\subsection*{Es. 2}

L'algoritmo iterativo � implementato nella classe \cpp{template} \cpp{SolveFP}:
\lstset{basicstyle=\scriptsize\sf}
\lstinputlisting[caption=Classe per l'implementazione di un algoritmo
iterativo di punto fisso]{es/2/src/fpsolve.hpp}
\lstset{basicstyle=\sf}

Si noti che il tipo di ritorno del metodo \cpp{iterate} � a sua
volta una classe: questo consente di esportare un insieme di
informazioni al programma chiamante.

La \emph{policy} per il parametro \cpp{template} \cpp{FP} � la
disponibilit� di un metodo pubblico \cpp{apply}: coerentemente con
questa richiesta viene implementata la classe astratta
\cpp{FixedPointBase}, classe base per una gerarchia di implementazioni
di funzioni di iterazione $\phi$.
\lstset{basicstyle=\scriptsize\sf}
\lstinputlisting[caption=Classe astratta per l'implementazione di una
funzione di iterazione $\phi$]{es/2/src/fpbase.hpp}
\lstset{basicstyle=\sf}

La funzione di iterazione di Newton viene implementata nella classe
\cpp{Newton}, derivata da \cpp{FixedPointBase}:
\lstset{basicstyle=\scriptsize\sf}
\lstinputlisting[caption=Classe \cpp{Newton}. Dichiarazione.]{es/2/src/fpnewton.hpp}
\lstinputlisting[caption=Classe \cpp{Newton}. Definizione.]{es/2/src/fpnewton.cpp}
\lstset{basicstyle=\sf}

Si riporta di seguito il \texttt{main
  program} che risolve l'esercizio. Anche in questo caso,
programmazione generica rispetto ai tipi e ereditariet� consentono di
realizzare strutture molto potenti e flessibili (grazie alla
modularit�). Si noti in particolare che l'introduzione di ulteriori
implementazioni di funzioni di iterazione � immediata.

\lstset{basicstyle=\scriptsize\sf}
\lstinputlisting[caption=\texttt{main program} per l'esercizio 2.]{es/2/main.cpp}
\lstset{basicstyle=\sf}


