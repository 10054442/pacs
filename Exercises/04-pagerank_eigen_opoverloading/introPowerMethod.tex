\subsection*{Power method}

The \emph{power method} can be applyed to matrices in which the maximum modulus
eigenvalue $\lambda_1$ has molteplicity one and it is well separated from the
closest eigenvalue with a smaller modulus. Here follows the generic step $k$ of
the algorithm
\begin{align*}
    {\bf q}\iter{k} &= \frac{ A{\bf q}\iter{k-1} }{
        \norm{ A{\bf q}\iter{k-1}}_2 } \\
    {\bf \nu}\iter{k} &= {{\bf q}\iter{k}}^T  A {\bf q}\iter{k} \\
    {\bf w}\iter{k} &= \frac{ {{\bf w}\iter{k-1}}^T A }{
        \norm{ {{\bf w}\iter{k-1}}^T A}_2 }
\end{align*}
where $\nu\iter{k}$, ${\bf q}\iter{k}$ and ${\bf w}\iter{k}$ are the
approximation of $\lambda_1$ and the aproximations of the left and right
eigenvector ${\bf x}_1$ e ${\bf y}_1$ associated to it. The following estimate
holds
\begin{align*}
    \module{ \lambda_1 - \nu\iter{k} } \approx %
    \frac{ \norm{ {\bf r}\iter{k} }_2 }{
    \module{ {{\bf w}\iter{k}}^T \cdot{\bf q}\iter{k} } }
\end{align*}
where ${\bf r}\iter{k} \eqbydef A{\bf q}\iter{k} - \nu\iter{k}{\bf
q}\iter{k}$ is the residual at iteration $k$. The estimate can be used to
implement a convergence criterion.
