\subsection*{Soluzione}

Viene fornita di seguito un'implementazione del metodo delle potenze
che sfrutta i vettori e le matrici fornite da Eigen.

La classe \cpp{LinearAlgebra::Eigenvalues::Power} possiede un unico
costruttore cui viene passata la tolleranza ed il massimo numero di
iterazioni. La ragione per cui si impone un limite al numero di
iterazioni \`e evitare che, in caso di mancata convergenza, il sistema
entri in un ciclo infinito. Il metodo viene applicato tramite una
chiamata al membro \cpp{apply}, cui va passata la matrice di cui si
vuole calcolare l'autovalore di modulo massimo ed una stima iniziale
dell'autovettore destro ad esso associato. All'intero del codice, viene
utilizzata tale approssimazione anche per inizializzare il calcolo
dell'autovettore sinistro. Il codice relativo alla classe \cpp{Power}
\`e riportato di seguito.
%
\lstset{basicstyle=\scriptsize\sf}
    \lstinputlisting[caption=Interfaccia della classe
        \cpp{Power}]{./es/pagerank-eigen/src/power.hpp}
\lstset{basicstyle=\sf}

Si noti i due \cpp{typedef} definiti pubblici all'inizio della classe.
Tale definizione \`e molto utile, dato che dall'esterno posso utilizzare
tali \cpp{typedef}. Cambiando la definizione all'interno della classe
cambiano automaticamente in tutto il codice. \`E interessante confrontare
la diverse implementazione dei due metodi \cpp{tol()}, oppure \cpp{maxit()}.
La firma di tali metodi differisce unicamente ``dall'ultimo \cpp{const}'' e,
ovviamente, non dal valore di ritorno. Nel primo caso posso modificare esternamente
il valore di \cpp{M_tol}, oppure \cpp{M_maxit}, mentre nel secondo caso no.
\`E stato definito anche il costruttore di copia e l'operatore di assegnazione
come privati, in questo modo inibiamo la possibilit\`a di utilizzarle quelli
di default. L'implementazione di tali metodi \`e inutile.

\lstset{basicstyle=\scriptsize\sf}
    \lstinputlisting[caption=Implementazione della classe
        \cpp{Power}]{./es/pagerank-eigen/src/power.cpp}
\lstset{basicstyle=\sf}

All'inizio del metodo \cpp{apply()} \`e interessante notare l'utilizzo di due tipi
di costruttori per i vettori delle Eigen. Nel primo e nel terzo caso viene utilizzato
il costruttore di copia, mentre nel secondo e quarto caso viene utilizzato il costruttore
che genera il vettore vuoto lungo \cpp{N}.

Il listato contenente il \cpp{main} \`e riportato qui di seguito
\lstset{basicstyle=\scriptsize\sf}
    \lstinputlisting[caption=Interfaccia della classe
        \cpp{Power}]{./es/pagerank-eigen/eig.cpp}
\lstset{basicstyle=\sf}

Notiamo l'utilizzo dei \cpp{typedef} che `` derivano '' dalla classe \cpp{PowerMethod},
tali \cpp{typedef} vengono visti dall'esterno come se la classe \cpp{PowerMethod} fosse
un \cpp{namespace}. Infatti \`e possibile accedere a tali elementi senza creare alcun
oggetto di classe \cpp{PowerMethod}. Tali \cpp{typedef} sono associati alla classe e non
alla particolare istanziazione della stessa.
