\section*{Esercizio 1}

Partendo dal programma \textbf{inherit1}
\begin{enumerate}
    \item estendere l'utilizzo di GetPot utilizzando un file per gestire
        i dati di input;

    \item si aggiunga la regola di quadratura gaussiana a quattro nodi,
        come classe derivata pubblicamente da \cpp{QuadratureRule}.
        Rispetto all'intervallo di riferimento $[-1,1]$ i nodi della
        regola di quadratura sono $\{ \pm 1, \pm \sqrt{5} / 5 \}$, e
        i rispettivi pesi sono $\{ 1/6, 5/6 \}$;

    \item si calcoli l'integrale della funzione
        \begin{align*}
            f(x) = e^x x\,,
        \end{align*}
        nell'intervallo $[0,1]$, utilizzando le tre regole
        di quadratura implementate.

    \item si utilizzi la tecnica del polimorfismo per gestire le
        istanze degli oggetti delle classi derivate da
        \cpp{QuadratureRule}.

    \item estrarre dalla classe \cpp{NumericalQuadrature} la
        suddivisione in intervalli, in modo che la generazione di
        intervalli sia a tutti gli effetti estranea alla classe.

        Si consiglia di
        \begin{itemize}
            \item costruire la classe \cpp{Domain1D} che implementa
                l'intervallo su cui fare l'integrazione;

            \item costruire la classe \cpp{Mesh1D} che contiene un
                oggetto di tipo \cpp{Domain1D}, all'interno della
                classe \cpp{Mesh1D} \`e generata la mesh equi-spaziata
                su cui fare l'integrale;

            \item la classe \cpp{NumericalQuadrature} conterr\`a la
                mesh e la regola di quadratura.

        \end{itemize}

    \item riorganizzare il codice in modo tale che le classi che
        necessitano dei dati definiti dall'utente accedano
        autonomamente ad essi. Ogni classe deve poter accedere
        unicamente alla propria sezione, senza tener conto delle sezioni
        superiori.

\end{enumerate}
