\documentclass[10pt,a4paper]{article}
\usepackage[T1]{fontenc}
\usepackage[a4paper]{geometry}
\usepackage{xcolor}
\usepackage{amssymb}
\usepackage{amsmath}
\usepackage{graphicx}
\usepackage{tabularx}
\usepackage{multirow}
\usepackage{subfigure}
\usepackage{verbatim}
\usepackage{fancyhdr}
\usepackage{listings}
\usepackage{../common/espacs}

%\input{../common/commands.tex}

\newcommand*{\vect}[1]{\boldsymbol{#1}}
\newcommand*{\mat}[1]{\boldsymbol{#1}}

\title{Exercise Session 5}
\date{November 9, 2012}

\pagestyle{fancy}
\headheight 35pt

\begin{document}
\lstset{language=[ISO]C++}
\maketitle

\section*{Exercise}

Starting from the given set of files that implement an algorithm that computes
the centroids for a clustering of 2D points with respect to the euclidean
distance

\begin{enumerate}

\item Make the object type a parameter of the class \cpp{Clustering}.

\item use the random generator to set the initial values of the centroids.

\item Develop a new \cpp{Distance} type called \cpp{AnisotropicDistance}, that
implements a non-isotropic metric $\mat{M}$ (an SPD matrix) in the 2D space. In
this case, the centroid $\vect{c}$ of a cluster satisfies
\[
\min_{\vect{c}} \sum_{i=0}^N d_{\mat{M}}( \vect{c}, \vect{p}_i )
\]
where $d_{\mat{M}}( \vect{a}, \vect{b} ) = \vect{a}^T \mat{M} \vect{b}$ is
the distance in the metric set by $\mat{M}$ and $N$ is the number of objects
$\vect{p}_i$ that have been assigned to the cluster. The minimization leads to
the linear system
\[
\mat{M} \vect{c} = \frac{1}{N} \sum_{i=0}^N \mat{M} \vect{p}_i
\]

\item Test the implementation using the metric
\[
\mat{M} =
\begin{pmatrix}
3 & 1 \\ 1 & 3
\end{pmatrix}
\]

\item Implement the \cpp{computeQuality} member function that gives an estimate
of the quality of the clustering based on the average distance of the objects
in the cluster from the centroid. Compare the value of the quality increasing
the number of clusters.

\end{enumerate}

%\subsection*{Exercise 1}

Build the connection matrix $B\in \mathbb{R}^{5 \times 5}$ relative to the web
made of the 5 pages, as depicted in Fig.~\ref{fig:web}.
%
\begin{figure}
\centering
\includegraphics[width=0.4\textwidth]{fig/tikz/pagerank5}
\caption{Scheme for a 5 pages web. Every circle is a page, every arrow is a
link.}
\label{fig:web}
\end{figure}
%
Using the \texttt{Eigen} library, write down a class that computes the maximum
eigenvalue (that is 1) and the corresponding eigenvector, that is the
\emph{pagerank}. Implement also \cpp{operator<<} to see the data inside the
class on the screen.

%\subsection*{Solution}

The proposed solution implements the power method using vectors and matrices
from the \texttt{eigen} library.

The class \cpp{LinearAlgebra::Eigenvalues::PowerMethod} has one constructor
that takes as parameters the tolerance and the max iteration number. We set
a max iteration number in order to avoid an infinite loop when there is no
convergence. The power method is applied with a call to the \cpp{apply} method,
that requires the matrix and an approximation of the right eigenvector as
input. The code uses this approximation also as seed for the left eigenvector.
The code for the \cpp{Power} class is as follows
%
\lstset{basicstyle=\scriptsize\sf}
    \lstinputlisting[caption=\cpp{Power} class interface]
    {./src/pagerank-eigen/src/power.hpp}
\lstset{basicstyle=\sf}

Note the two public \cpp{typedef}s at the beginning of the class. They can be
very useful from outside the class, since changing them inside the class updates
automatically all the code. It is interesting to compare the two implementations
for the \cpp{tol} and \cpp{maxit} methods. When there is a \cpp{const} keyword
at the end of the definition, it means that the method is intended to be just
for access. When there is no \cpp{const} keyword, the method is instead used to
modify the value of the private member. The copy constructor and the assignment
operator have the keyword \cpp{delete}, this means that the compiler is now
allowed to instantiate them implicitly. This way of disabling implicit
implementations is only available with \texttt{c++11}. If the new standard is
not available, a common way to avoid implicit instantiations is to declare the
method in the private section of the class and not implement it. this will raise
a compile error if the method is requested somewhere.

\lstset{basicstyle=\scriptsize\sf}
    \lstinputlisting[caption=\cpp{Power} class implementation]
    {./src/pagerank-eigen/src/power.cpp}
\lstset{basicstyle=\sf}

At the beginning of the \cpp{apply} method it is interesting to note the use of
two different constructors for \texttt{eigen} vectors. In the first and third
case the copy constructor is used, while on the second and forth case we use the
constructor that builds an empty vector of size \cpp{N}.

The listing for the \cpp{main} program follows
\lstset{basicstyle=\scriptsize\sf}
    \lstinputlisting[caption=Main program]
    {./src/pagerank-eigen/eig.cpp}
\lstset{basicstyle=\sf}

We note the use of the \cpp{typedef}s that come from the \cpp{PowerMethod}
class. They are accessed from outside the class with their fully specialized
name, with a syntax similar to the one used for namespaces. It is worth noting
that the use of the typedefs does not require the creation of an object of type
\cpp{PowerMethod}, just like the use of static functions and members does not
too.



\end{document}
