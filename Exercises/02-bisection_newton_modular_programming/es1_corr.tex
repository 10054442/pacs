\subsection*{Soluzione es. 1}

La soluzione dell'esercizio \`e riportata nel seguente listato:
%
\lstset{basicstyle=\scriptsize\sf}
    \lstinputlisting{./es/1/all_in_one/bn-allinone.cpp}
\lstset{basicstyle=\sf}

Si noti, innanzitutto, che \`e stato definito un tipo utente
\cpp{real}. Qualora si volesse modificare la precisione
con cui si effettuano i calcoli, sarebbe sufficiente cambiare solo la
linea contenente la definizione del tipo \cpp{real}. La modifica si
propagherebbe in modo coerente in tutto il codice.

Per rendere l'implementazione dei metodi di ricerca indipendente dalla
funzione considerata si \`e adottata la tecnica dei \emph{puntatori a
funzione}. L'istruzione:
\begin{lstlisting}
typedef real (*fctptr)(real);
\end{lstlisting}
definisce il tipo \cpp{fctptr} come un puntatore a funzioni che
ricevono un argomento di tipo \cpp{real} e restituiscono un valore di
tipo \cpp{real}.

L'arresto dei metodi iterativi per la ricerca degli zeri di una
funzione pu\`o avvenire in base a diversi criteri. Per funzioni che
abbiano derivata prima vicina a $1$ in modulo in corrispondenza dello
zero cercato un efficiente metodo d'arresto \`e basato sul controllo
del \emph{residuo}. In tal caso si considera raggiunta la convergenza
alla prima iterazione $k$ in cui
$\module{\function{f}{x\iter{k}}}<\cpp{tol}$. Un metodo alternativo
basato sul controllo dell'\emph{incremento} risulta, per contro,
efficiente qualora si utilizzino delle iterazioni di punto fisso e la
funzione di iterazione $\function{\phi}{\cdot}$ abbia derivata prima
lontana da $1$ in modulo in corrispondenza dello zero cercato. Tale
criterio prevede l'arresto alla prima iterazione $k$ per cui si abbia
$\module{x\iter{k+1} - x\iter{k}}<\cpp{tol}$. Nel caso del metodo di
Netwon, in cui, per definizione, $\function{\phi'}{\alpha} = 0$, il
controllo dell'incremento fornisce un bilanciamento
ottimale. Nell'implementazione si \`e tenuto conto della
molteplicit\`a dei criteri d'arresto definendo un tipo enumerativo
\cpp{checkT} e dotando ogni metodo di una propriet\`a \cpp{M\_check}
di tipo \cpp{checkT} che specificasse il metodo da utilizzare.

Compilando il codice con i comandi
\begin{verbatim}
    g++ -o bn-allinone bn-allinone.cpp -Wall
\end{verbatim}
Ed eseguendolo, otteniamo
\begin{verbatim}
0.707107        27
0.707107        7
0.707107        14 1
\end{verbatim}
