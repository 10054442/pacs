\subsection*{Note sui cicli}

Il codice proposto al \emph{Punto 1} utilizza un ciclo \cpp{while}.
%Un' altra alternativa possibile \`e l'uso di un ciclo \texttt{while}:
%\lstset{basicstyle=\scriptsize\sf}
%\lstinputlisting[caption=Ciclo \texttt{while} senza
%\texttt{break}.,linerange={26-47}]{ex/1/zerofun-while.cc}
%\lstset{basicstyle=\sf}
In queso caso la prima valutazione della funzione \`e necessariamente esterna
al ciclo.

In alternativa si potrebbe scrivere un ciclo \cpp{for} ed utilizzare
l'istruzione \texttt{break} qualora venisse raggiunta la convergenza entro
il massimo numero di iterazioni.
\lstset{basicstyle=\scriptsize\sf}
    \lstinputlisting[caption=Ciclo \texttt{for} con
        \texttt{break}.,linerange={26-67}]{ex/1/old_file/zerofun-break.cc}
\lstset{basicstyle=\sf}
L'istruzione \texttt{break} \`e un modo efficiente per
imporre una condizione di uscita, ma pu\`o rendere il codice di difficile
lettura ed interpretazione; infatti le condizioni di uscita sono sparse e non
sono raggruppate all'inizio o alla fine. Per questo motivo si cerca di limitare
l'uso di \texttt{break} alla gestione di eccezioni.

Una soluzione alternativa \`e basata sul ciclo \cpp{do}/\cpp{while}:
\lstset{basicstyle=\scriptsize\sf}
\lstinputlisting[caption=Ciclo \texttt{do\ldots while} senza
    \texttt{break}.,linerange={26-47}]{ex/1/old_file/zerofun-dowhile.cc}
\lstset{basicstyle=\sf}
In questo caso le condizioni di uscita sono chiaramente visibili in fondo al
codice. Il prezzo da pagare \`e un inutile assegnamento di variabili
nell'ultima iterazione eseguita. Il costrutto \texttt{do \ldots while} ha come
caratteristica fondamentale quella di eseguire sempre almeno una iterazione,
anche se le condizioni di uscita non sono verificate o
verificabili all'inzio del ciclo.
A volte questo comportamento pu\`o aiutare a generare errori non banali,
se il codice scritto \`e complesso.

