\section*{Es. 2}

Considerando quanto fatto nell'esercizio 1, si richiedere di:
\begin{enumerate}

    \item suddividere il programma in pi\`u file, in maniera modulare,
        cos\`i da favorire la riusabilit\`a del codice.
        Si suggerisce di raggruppare tutte le implementazioni delle funzioni
        in un file, scrivere un corrispondente file di intestazione
        (\emph{header}) e lasciare \cpp{main} in un terzo file.

% \item implementare un metodo per la ricerca dello zero di $f$
%   sfruttando l'algoritmo della bisezione per avvicinarsi alla radice
%   ed il metodo di Newton per ottenerne una stima accurata (si rimanda
%   a \cite{Quarteroni.Sacco.ea:2000} per l'analisi).

    \item ripetere l'ultimo punto dell'esercizio 1;

    \item sapendo che la soluzione esatta del punto precedente \`e
        $x=\sqrt{0.5}$, modificare il codice in modo tale da poter salvare
        la distanza tra la soluzione approssimata e quella approssimata
        alla corrente iterazione in un file.
        Visualizzare la soluzione tramite Gnuplot.
        Commentare il risultato alla luce della teoria.

% Si verifichi che, scegliendo come
%   punto di partenza $x^{\left(0\right)} = 0$, il metodo di Netwon non
%   converga.
%\item Sovraccaricare l'operatore di scorrimento in modo che, inviando
%  ad uno \emph{stream} la classe corrispondente a ciascun metodo, si
%  ottenga una descrizione dell'istanza (nome del metodo, valori dei
%  parametri, numero di iterazioni necessarie per raggiungere la
%  convergenza all'ultima chiamata).
\end{enumerate}
