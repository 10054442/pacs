\documentclass{beamer}

%\usepackage{t1enc}
%\pgfpagelayout{2 on 1}[a4paper]

\usepackage{beamerthemesplit,bm}
\usepackage[latin1]{inputenc}
\usepackage[italian]{babel}
\usepackage{graphicx}
% \usepackage{movie15}
\usepackage{hyperref}
\usepackage{multimedia}
\usepackage{subfigure}
\usepackage{xcolor}
\usepackage{amsmath,amssymb}
\usepackage{stmaryrd}

\usetheme{Boadilla}


\definecolor{mygreen}{rgb}{0,0.48,0.0}

\definecolor{myblue}{rgb}{0,0,0.64}

\author{Alessio Fumagalli}
\date{20-10-2011}
\institute{Politecnico di Milano}

\begin{document}

%---------------------------------------------------------------------------------

\begin{frame}[fragile]

    \frametitle{Gestione dei file testuali}

    \begin{block}{Headerfile da aggiungere}
        \begin{verbatim}
#include <fstream>
        \end{verbatim}
        Utile sia per l'apertura del file in lettura che in scrittura.
    \end{block}

\end{frame}

%---------------------------------------------------------------------------------

\begin{frame}[fragile]

    \frametitle{Gestione dei file testuali}

    \begin{block}{Dichiarazione e apertura di un file in scrittura}
        \begin{verbatim}
std::fstream fileOut;
fileOut.open ( fileName.c_str(), std::fstream::out |
                                 std::fstream::app );
        \end{verbatim}
    \end{block}

    \begin{block}{Dichiarazione e apertura di un file in lettura}
        \begin{verbatim}
std::fstream fileIn;
fileIn.open ( fileName.c_str(), std::fstream::in );
        \end{verbatim}
    \end{block}

\end{frame}

%---------------------------------------------------------------------------------

\begin{frame}[fragile]

    \frametitle{Gestione dei file testuali}

    \begin{block}{Controllare che il file sia aperto}
        \begin{verbatim}
if ( file.is_open() == false )
{
    std::cerr << "File not opened" << std::endl;
    exit(1);
}
        \end{verbatim}
    \end{block}

\end{frame}

%---------------------------------------------------------------------------------

\begin{frame}[fragile]

    \frametitle{Gestione dei file testuali}

    \begin{block}{Scrittura su file}
        \begin{verbatim}
fileOut << "Ciao " << 5.3 << std::endl;
        \end{verbatim}
    \end{block}

    \begin{block}{Lettura da file}
        \begin{verbatim}
char name [ 256 ];
float value;
fileIn >> value;
fileIn.getline ( name, 256, '\n' );
        \end{verbatim}
    \end{block}

\end{frame}

%---------------------------------------------------------------------------------

\begin{frame}[fragile]

    \frametitle{Gestione dei file testuali}

    \begin{block}{Ricordarsi di chiudere il file}
        \begin{verbatim}
file.close();
        \end{verbatim}
    \end{block}

\end{frame}

%---------------------------------------------------------------------------------

\begin{frame}[fragile]

    \frametitle{Gestione dei file testuali}

    \begin{block}{Scrivere numeri reali su file}
        Per aggiungere cifre significative di un numero reale
        nella scrittura in un file occorre aggiungere
        \begin{verbatim}
fileOut.precision ( 15 );
        \end{verbatim}
    \end{block}

\end{frame}

%---------------------------------------------------------------------------------

\end{document}
